\documentclass[12pt]{article}

\usepackage{sbc-template}

\usepackage{graphicx,url}

\usepackage[brazil]{babel}   
%\usepackage[latin1]{inputenc}  
\usepackage[utf8]{inputenc}  
% UTF-8 encoding is recommended by ShareLaTex

\usepackage{url}
\usepackage{cite}
\usepackage{hyperref}

     
\sloppy

\title{Uso de Projetos de Código Aberto 
	\\no Ensino de Inteligência Artificial}

\author{Esdras L. Bispo Jr.\inst{1} }

\address{Instituto de Ciências Exatas (ICET), Regional Jataí\\ Universidade Federal de Goiás (UFG)\\
  BR 364, km 195, nº 3800, CEP 75801-615 -- Jataí -- GO -- Brasil
  \email{bispojr@ufg.br}
}

\begin{document} 

\maketitle

\begin{abstract}
  Something...
\end{abstract}
     
\begin{resumo} 
  Algo...
\end{resumo}


\section{Introdução}

A aprendizagem por projetos consiste em propor aos alunos cenários semelhantes aos do mundo real e conduzi-los através da construção de uma solução possível. A aprendizagem por projetos está fortemente associada aos conceitos de aprendizagem pela prática ({\it learning by doing}) \cite{anzai:1979, schank:1999, benssen:2015}, aprendizagem autêntica \cite{herrington:2000, herrington:2006, lombardi:2007} e educação direta \cite{lakey:2010}. Este trabalho tem como o objetivo apresentar como o uso do desenvolvimento de software livre contribui para o ensino de Inteligência Artificial (IA).

O Bispix\footnote{\url{http://www.github.com/freeufg/bispix}} é um software livre que foi criado para a realização de projetos na disciplina de IA. O propósito é de que os alunos desenvolvam extensões do Bispix utilizando o modelo de ciclo de vida de software {\it Fork and Pull} \cite{alasbali:2015, buffardi:2015}. Todo o processo é realizado através do GitHub,  permitindo aos alunos o desenvolvimento a partir de um código pré-existente (algo bastante comum no mundo real).

Será apresentado em mais detalhes a seguir alguns conceitos importantes como a aprendizagem pela prática (Seção \ref{sec:doing}), aprendizagem autêntica (Seção \ref{sec:authentic}) e educação direta (Seção \ref{sec:direct}). Logo após, apresentamos o Bispix (Seção \ref{sec:bispix}), e como ele está estruturado. Por fim, apresentamos algumas perspectivas de como gerar resultados de pesquisas interessantes com esta iniciativa (Seção \ref{sec:perspectivas}).

\section{Teoria da Aprendizagem pelo Fazer} \label{sec:doing}

Um experimento foi realizado com uma japonesa durante uma hora e meia. A missão dela era resolver o problema da Torre de Hanói com cinco discos. Era necessário que ela não apenas resolvesse o problema, mas o fizesse da melhor forma possível.

Anzai e Simon, os pesquisadores, pediram que a japonesa falasse em voz alta o que ela estivesse pensando durante todo o processo \cite{anzai:1979}. O propósito deles era identificar se havia algum padrão nas estratégias de resolução dela. Este experimento serviu como base para a Teoria da Aprendizagem pelo Fazer\footnote{Do inglês, {\it Theory of Learning by Doing}} (TAF).

O TAF identifica os processos que habilitam um estudante a aprender enquanto está engajado na resolução de um problema.

aprendizagem pela prática ({\it learning by doing}) \cite{schank:1999, benssen:2015}

\section{Aprendizagem Autêntica} \label{sec:authentic}
\section{Educação Direta} \label{sec:direct}
\section{Bispix} \label{sec:bispix}
\section{Perspectivas de Pesquisa} \label{sec:perspectivas}


\bibliographystyle{sbc}
\bibliography{bibliografia}

\end{document}
